\documentclass[]{article}
\usepackage{lmodern}
\usepackage{amssymb,amsmath}
\usepackage{ifxetex,ifluatex}
\usepackage{fixltx2e} % provides \textsubscript
\ifnum 0\ifxetex 1\fi\ifluatex 1\fi=0 % if pdftex
  \usepackage[T1]{fontenc}
  \usepackage[utf8]{inputenc}
\else % if luatex or xelatex
  \ifxetex
    \usepackage{mathspec}
  \else
    \usepackage{fontspec}
  \fi
  \defaultfontfeatures{Ligatures=TeX,Scale=MatchLowercase}
\fi
% use upquote if available, for straight quotes in verbatim environments
\IfFileExists{upquote.sty}{\usepackage{upquote}}{}
% use microtype if available
\IfFileExists{microtype.sty}{%
\usepackage{microtype}
\UseMicrotypeSet[protrusion]{basicmath} % disable protrusion for tt fonts
}{}
\usepackage{hyperref}
\hypersetup{unicode=true,
            pdfborder={0 0 0},
            breaklinks=true}
\urlstyle{same}  % don't use monospace font for urls
\IfFileExists{parskip.sty}{%
\usepackage{parskip}
}{% else
\setlength{\parindent}{0pt}
\setlength{\parskip}{6pt plus 2pt minus 1pt}
}
\setlength{\emergencystretch}{3em}  % prevent overfull lines
\providecommand{\tightlist}{%
  \setlength{\itemsep}{0pt}\setlength{\parskip}{0pt}}
\setcounter{secnumdepth}{0}
% Redefines (sub)paragraphs to behave more like sections
\ifx\paragraph\undefined\else
\let\oldparagraph\paragraph
\renewcommand{\paragraph}[1]{\oldparagraph{#1}\mbox{}}
\fi
\ifx\subparagraph\undefined\else
\let\oldsubparagraph\subparagraph
\renewcommand{\subparagraph}[1]{\oldsubparagraph{#1}\mbox{}}
\fi
			MathJax.Hub.Config({
				showProcessingMessages: false,
				messageStyle: "none",
				TeX: { equationNumbers: {autoNumber: "all"} },
				extensions: ["tex2jax.js"],
				jax: ["input/TeX", "output/HTML-CSS"],
				tex2jax: {
					inlineMath: [ ['$','$'] ],
					displayMath: [ ['$$','$$'] ],
					processEscapes: true
				},
		        SVG: { linebreaks: { automatic: true } },
				"HTML-CSS": { availableFonts: ["TeX"] }
			});
/*
			MathJax.Hub.Config({
		        TeX: { noErrors: { disabled: true } }
			});
*/

\date{}

\begin{document}

\section{Calculating the magnetic field with the Biot-Savart
Law}\label{calculating-the-magnetic-field-with-the-biot-savart-law}

Developed by J. D. McDonnell

In this set of exercises, the student will implement code for the
Biot-Savart law to compute the magnetic field at any point in space due
to a square-shaped loop.

Upon calculating the magnetic field at many points in space, the student
will prepare a vector plot of the magnetic field.

\subsection{Exercises}\label{exercises}

\subsubsection{Exercise 1: Calculate the magnetic field due to a
straight segment of
wire}\label{exercise-1-calculate-the-magnetic-field-due-to-a-straight-segment-of-wire}

Consider a straight segment of wire \(1\) unit long. Place one end at
\((0.5, 0.5, 0.0)\), and the other end at \((-0.5, 0.5, 0.0)\). Let a
steady current of \(1\)A flow through this wire, from the first end
towards the second end. (Such a current cannot exist physically, but
this is still a good first-step towards the goal of these Exercises.)

\begin{enumerate}
\def\labelenumi{\arabic{enumi}.}
\tightlist
\item
  Use the Biot-Savart law to analytically calculate the magnetic field
  at the origin. This will serve as a check for the numerical method.\\
\item
  Describe in words (or pseudocode) a procedure to \emph{numerically}
  calculate the magnetic field at the origin with the Biot-Savart law.\\
\item
  Implement the code to numerically calculate the magnetic field at the
  origin with the Biot-Savart law. Make sure that your numerical answer
  agrees with the analytical answer.
\end{enumerate}

\subsubsection{Exercise 2: Calculate the magnetic field at the center of
a square
loop}\label{exercise-2-calculate-the-magnetic-field-at-the-center-of-a-square-loop}

Consider a square loop of wire, sitting in the \(xy\)-plane. There is a
current of \(1\)A flowing through the wire. The goal of these exercises
is to calculate the magnetic field that results from this current
configuration.

\begin{enumerate}
\def\labelenumi{\arabic{enumi}.}
\tightlist
\item
  Use the Biot-Savart law to analytically calculate the magnetic field
  at the center of the square loop - assume the loop has sides of length
  \(1\) unit for simplicity. This will serve as a check for the
  numerical method. \textbf{Note}: For both this analytical calculation
  and your numerical calculation below, think of this square loop of
  wire as four segments of straight wire connected to each other. You
  can then build from your work in Exercise 1.\\
\item
  Describe in words (or pseudocode) a procedure to \emph{numerically}
  calculate the magnetic field at the center of the loop with the
  Biot-Savart law.\\
\item
  Implement the code to numerically calculate the magnetic field at the
  center of the loop with the Biot-Savart law. Make sure that your
  numerical answer agrees with the analytical answer.
\end{enumerate}

\subsubsection{Exercise 3: Calculate the magnetic field of a square loop
at any point in
space}\label{exercise-3-calculate-the-magnetic-field-of-a-square-loop-at-any-point-in-space}

Now that you have validated your numerical approach for the magnetic
field at the center of the loop, your new task is to \emph{generalize}
your approach in order to calculate the magnetic field at \emph{any}
point in space \(\vec{r}\). In particular, you will calculate the
magnetic field at many grid points in the \(yz\)-plane, and from there
you will be able to visualize the magnetic field with a vector plot.

\begin{enumerate}
\def\labelenumi{\arabic{enumi}.}
\tightlist
\item
  Describe in words (or pseudocode) the modifications you will need to
  make to your previous procedure. Discuss the way that you will
  calculate the magnetic field at many points in the \(yz\)-plane.
\item
  Implement the code to calculate the magnetic field at many points in
  the \(yz\)-plane.
\item
  From the magnetic field that you have calculated, produce a vector
  plot. Describe the key features that you see. Does the image match
  your expectations? Is there anything that surprises you? Where is the
  magnetic field the strongest, and where is it weakest? How can you
  tell from the field lines?
\end{enumerate}

\subsubsection{Extension: Calculate the magnetic field for any wire
shape}\label{extension-calculate-the-magnetic-field-for-any-wire-shape}

The first two exercises focus on calculating the magnetic field due to a
square loop of current-carrying wire. It is natural to extend this
exercise to calculate the magnetic field due to \emph{other} interesting
shapes of curent-carrying wire. Some suggested shapes:

\begin{itemize}
\tightlist
\item
  A circular loop of wire. In this case, you will be able to perform an
  analytical verification of the magnetic field at the center of the
  loop again.\\
\item
  A helix of wire. If you make the helix long enough and with tight
  ``windings'', you can approximate a finite-length solenoid.
\end{itemize}

\end{document}
